\documentclass[11pt,a4paper]{article}

\usepackage{fancyhdr}
\usepackage{float}
\pagestyle{fancy}
\setlength{\parskip}{3pt}
\usepackage{amsfonts}
\usepackage{amsmath}
\usepackage{amsthm}
\usepackage{amssymb}
\usepackage[numbers]{natbib}
\usepackage[pdftex]{graphicx}
\usepackage{asymptote}
\usepackage[final]{pdfpages}


%\usepackage[all]{xy}
\newcommand{\comment}[1]{}
%\usepackage[margin=1in]{geometry}        

%\setlength{\oddsidemargin}{0mm}
%\setlength{\evensidemargin}{0mm}
%\setlength{\topmargin}{0mm}
%\setlength{\headheight}{0mm}
%\setlength{\headsep}{0mm}
%\setlength{\textheight}{227mm}
%\setlength{\textwidth}{158mm}
%\setlength{\voffset}{-.5in}
%A4 = 210 mm x 297 mm

\setlength{\headheight}{26mm}
\setlength{\oddsidemargin}{0in}
\setlength{\evensidemargin}{0in}
\setlength{\voffset}{-36mm}
\setlength{\headsep}{10pt}
\setlength{\textwidth}{158mm}
\setlength{\headwidth}{158mm}
\setlength{\textheight}{245mm}
\setlength{\parindent}{0.0in}
\setlength{\parskip}{0.1in}

\newtheorem{thm}{Theorem}[section] 
\theoremstyle{definition}
\newtheorem{lem}[thm]{Lemma}
\newtheorem{conj}[thm]{Conjecture}
\newtheorem{claim}[thm]{Claim}
\newtheorem{mydef}[thm]{Definition}
\newtheorem{prop}[thm]{Proposition}



\begin{document}
\title{\sc Hirsch Conjecture and the Simplex Algorithm}
\author{\sc Sunil Abraham\\
\sc St. John's College\\
\\
\sc 3320 Thornhill Dr.\\
\sc Adrian, MI 49221\\
\sc USA}
\date{}
\maketitle


I declare that this essay is work done as part of the Part III Examination. I have
read and understood the Statement on Plagiarism for Part III and Graduate Courses
issued by the Faculty of Mathematics, and have abided by it. This essay is the result
of my own work, and except where explicitly stated otherwise, only includes material
undertaken since the publication of the list of essay titles, and includes nothing
which was performed in collaboration. No part of this essay has been submitted, or
is concurrently being submitted, for any degree, diploma or similar qualification at
any university or similar institution.
\\
\\
Signed:\dotfill\today
\newpage
\tableofcontents
\newpage
\section{Introduction}
\begin{quote}
\emph{The final test of a theory is its capacity to solve the problems which originated it.}\\
--- George Dantzig
\end{quote}

Consider the general problem of minimizing a linear objective function, in $n$ variables, subject to $m$  linear equality constraints. We express this as the primal problem,
$$\text{(P): minimize}\left\{ c^T x : Ax= b, x\ge 0\right\}$$
where $A\in\mathbb{R}^{m\times n}$, $b\in\mathbb{R}^m$, and $c\in\mathbb{R}^n$. We assume $A$ is of full row rank, $m$, and that (P) has a feasible solution. We can then express the feasible region of (P) geometrically as the intersection of an $n-m$ dimensional affine subspace, as defined by $Ax\le b$, with the positive orthant, as $x\ge 0$. This is the immediate geometric interpretation, but interestingly Dantzig's reformulation of the feasible set as the convex hull of points $A_j=(a_j,c_j)$, where $a_j$ is the $j$th column of $A$ and $c_j$ is the $j$th component of $c$, led him to reconsider the \emph{simplex algorithm} as an efficient solver of (P) \citep[Chapter 7]{dant63}. For now, let us satisfy ourselves with basics, first a definition of the \emph{feasibility polyhedron, P}:
$$P:=\left\{x\in\mathbb{R}^n : Ax=b, x\ge0\right\}.$$
It may be useful at times to denote $P$ as
$$P=P(A,b).$$
Also for convenience, we say $P$ is in the \emph{class} $(d,n)$, if $P$ is a $d$-dimension polytope with $n$ facets. The triple $(P,x,y)$ consists of a polytope $P$ and vertices $x,y\in P$. $(P,x,y)$ is in the \emph{class} $(d,n)$ if $P$ is in the class $(d,n)$. 

By our assumption of $A$ being rank $m$, we have that $P$ is a polyhedron with vertices. We abuse standard notation here a little, as a polyhedron has extreme points, but the polyhedron's graph has vertices, which naturally correspond to the polyhedron's extreme points. For the sake of simplicity, the notion of a polytope and its graph are treated as the same in this paper. As we are trying to minimize a linear function over a affine space, we can limit our search to the space's extremities, the polyhedron's vertices. This is effectively what the simplex algorithm does. Each vertex of $P$ is a basic feasible solution, and the algorithm moves along adjacent vertices of $P$ until it strikes an optimal solution. The simplex algorithm's pivot rule decides which adjacent vertex to choose. The pivot rule is locally determined, by the adjacent vertices, not globally determined, by all of the polyhedron's vertices, so, a priori, we do not know how many pivots the algorithm will require. 

Most deterministic rules have been found to have a exponential worst-case complexity. In 1969, Klee and Minty were the first to give an example of such a case \citep{klee69}. For Dantzig's classic rule of most reduced cost per pivot, they prove that $2^m-1$ pivots may be required, where $m$ is both the number of inequality constraints and nonnegative variables. (As an aside, most sources indicate that Klee and Minty demonstrated this in 1972, but in fact they presented this result in 1969 at the Third Symposium on Inequalities. The confusion likely lies in the proceedings of the conference was published later, in 1972.) Somewhat surprisingly then, non-deterministic pivot rules have been discovered with sub-exponential, but super-polynomial, worst-case complexity. \citet{kalai92} and, later, independently, \citet{mato96} presented a randomized sub-exponential pivot rule and proved that the expected number of pivots and the expected number of arithmetic operations, both in the worst case, are bounded by $m^{O(\sqrt{n/\log n})}$ and $\exp(K\sqrt{n\log m})$, respectively, for (P) in $n$ variables and $m$ inequalities (not equalities). 

Nonetheless, linear programming is a class P problem. In 1979, \citet{khach79} demonstrated that the ellipsoid method of \citet{shor77} and \citet{judin76}, \citep{judin76b} solves (P) polynomially in its worst case. However, in practice, the ellipsoid method is rarely used; through observation, it became clear that variants of the simplex, albeit exponential, were very efficient in practice, whereas the ellipsoid method, though polynomial, was very inefficient for the same problem \citep{meg87}.

The simplex algorithm typically takes between $m$ to $3m$ pivots to terminate, for an $m$-equation problem with $m$ different variables in the final basic set. This empirical observation was made as early as 1963 by \citet{dant63} and \citet{wolfe63}. 

Efforts to explain the success of the simplex algorithm have looked at the \emph{combinatorial diameter} of a polytope, $P$, which we define as
$$\delta(P):= \text{the minimum number of edges in a path between any two vertices in G}$$
Further let us define
\begin{align*}
\Delta(d,n)&:= \text{the maximum diameter of a polytope in class $(d,n)$, and}\\
\Delta_u(d,n)u&:=\text{the maximum diameter of an \emph{unbounded} polytope in class $(d,n)$}.
\end{align*}
Then, if the simplex algorithm chose vertices optimally, but were initiated at the worst vertex, the algorithm would require $\Delta$ or $\Delta_u$ pivots. In the 1957, Hirsch conjectured to Dantzig through correspondence  that$\Delta(d,n)\le n-d$ and $\Delta_u(d,n)\le n-d$ \citep{dant63}. For this paper, we separate this into two cases:
\begin{conj}
	(Bounded Hirsch's Conjecture) $\Delta(d,n)\le n-d$.
\end{conj}
\begin{conj}
	(General Hirsch's Conjecture) $\Delta_u(d,n)\le n-d$.
\end{conj}
\begin{figure}
\begin{center}
\begin{asy}
	size(10cm);
	pen bg=gray(0.9)+opacity(0.5);
	path a=(0,0)--(1,-1)--(3,-2)--(5,-1.5)--(6,1)--(4,4)--(2,3)--cycle;
	filldraw(a,bg,black);
	dot((0,0)); dot((1,-1)); dot((3,-2));
	dot((5,-1.5),red); dot((6,1)); dot((4,4));
	dot((2,3),red);
	path b=(8,4)--(9,0)--(10,-1)--(12,-2)--(13,-1)--(15,4);
	path c=(8,4)--(9,0)--(10,-1)--(12,-2)--(13,-1)--(15,4)--cycle;
	filldraw(c,bg,white);
	draw(b);
	dot((8.5,2),red); dot((9,0)); dot((10,-1));
	dot((12,-2)); dot((13,-1)); dot((14,1.5),red);
\end{asy}
\caption{$\Delta(2,n)=\lfloor n/2\rfloor$ and $\Delta_u(2,n)=n-2$. The above is an example of $f=7$.}
\end{center}
\end{figure}
In relation to our interest in linear programming, Hirsch's conjecture is connected with the complexity of the simplex algorithm. In particular, it connects to the question of whether there can exist a pivot rule such that the simplex algorithm is a strongly polynomial algorithm for solving (P).

This essay provides an overview of Hirsch's Conjecture from its proposition over 50 years ago to current work presented earlier this year. In Section 2, we set review the developments that led to the disproof of Hirsch's Conjecture. Because we know the Hirsch Conjecture to be false, in Section 3, we highlight the best bounds on general polytope diameter that we do know. Particular polytopes such as the assignment polytope, network flow polytope, and shortest route polytopes are of particular interest to academics and practitioners, and because of these polytopes are highly structured, we know very good results for their diameters. We will discuss these specific cases in Section 4. Finally, we overview recent work and the direction of future work, particularly of the polymath 3 project.

\section{Disproving Hirsch's Conjecture}
\begin{quotation}
\emph{I will first examine the difference between imagination and pure understanding. When I imagine a triangle, for example, I do not merely understand that it is a figure bounded by three lines, but at the same time I also see the three lines with my mind's eye as if they were present before me; and this is what I call imagining. But if I want to think of a chiliagon, although I understand that it is a figure consisting of a thousand sides just as well as I understand the triangle to be a three-sided figure, I do not in the same way imagine the thousand sides or see them as if they were present before me...in doing this I notice quite clearly that imagination requires a peculiar effort of mind which is not required for understanding; this additional effort of mind clearly shows the difference between imagination and pure understanding.}\\
--- Ren\'{e} Descartes \citep{desc96}
\end{quotation}
We should admit that we now know that Hirsch's conjecture both for bounded and unbounded polytopes is false. However, how we came to know this is instructive. So we go through some of the major and interesting results that led to the disproof.

\begin{claim}To find $\Delta$, it is sufficient to consider simple polytopes.
\end{claim}
\begin{proof}
For any polytope in class $(d,n)$, we construct a simple polytope in class $(d,n)$ with diameter at least as large as the original's.

Consider a polytope $P=P(A,b)$ in class $(d,n)$. There exists $b'$, a perturbation of $b$, such that $P'=P(A,b')$ is simple. To make it explicit, let $b'_i=b_i+\epsilon^i$ and take $\epsilon$ to be small. By a natural bijection between $P$ and $P'$, $P'$ is then also a $(d,n)$ class polytope; however, by perturbing the hyperplanes, increased the number of vertices and reduced the number of facets adjacent to some vertices.

These new vertices cannot have reduced the polytope's diameter though; they were only added between vertices in $P$ that were adjacent to more than $d$ facets and their neighbors. Hence, to find $\Delta$, and prove Hirsch's conjecture, it is sufficient to consider simple polytopes.

Note: This claim is made in \citet{klee67} and \citet{zieg94} but no proof is given; this argument is the author's work, so any mistake is the author's alone.
\end{proof}
Working on simple polytopes, we have the following conjecture by \citet{klee65}:
\begin{conj}
	(Nonrevisiting path conjecture) Any two vertices of a simple polytope can be joined by a path that does not visit a facet more than once.
\end{conj}
Clearly, if the nonrevisiting path conjecture were true, then Hirsch's conjecture would be too. To see this, note that the initial vertex lies on $d$ facets, and each subsequent vertex in the nonrevisiting path is adjacent to at least one facet that has not been visited. Hence the path cannot be greater than $n-d$. Hence, we have that the nonrevisiting path conjecture implies Hirsch's conjecture.

A special case of Hirsch's conjecture is when $n=2d$:
\begin{conj}
	(Bounded $d$-step Conjecture) $\Delta(d,2d)\le d$.
\end{conj}
\begin{conj}
	(General $d$-step Conjecture) $\Delta_u(d,2d)\le d$.
\end{conj}
Surprisingly, the the bounded $d$-step conjecture and the nonrevisiting path conjecture are both equivalent to the bounded Hirsch conjecture \citep{klee67}.

\begin{thm}\label{thm-hirschequiv}
	The following are equivalent.
	\begin{enumerate}
		\item (Bounded Hirsch's Conjecture) $\Delta(d,n)\le n-d$.
		\item (Bounded $d$-step Conjecture) $\Delta(d,2d)\le d$.
		\item (Nonrevisiting path conjecture) Any two vertices of a simple polytope can be joined by a path that does not visit a facet more than once.
	\end{enumerate}
\end{thm}
To prove Theorem \ref{thm-hirschequiv} we introduce some definitions in the convention of \citet{klee67}.

The \emph{product} of $d_1$ dimensional polyhedron $P_1$ and $d_2$ dimensional polyhedron is $d_1+d_2$ dimensional polyhedron $P_1\times P_2$.

Now we define the \emph{wedge} of a $(d,n)$ polytope $P$ over foot $k$-face $K$. Take $C$ to be the product of $L$ and $P$, where $L=[0,\infty)$. Take $H$ to be a hyperplane in the space spanned by $C$ such that $K=H\cap P$ and $H$ is in the interior of $C$. $H$ divides $C$ into two $d+1$-dimensional polyhedra. By construction, only one of these polyhedra contains $P$. Take $W$ be the polyhedron that contains $K$. Now, $W$ is called the \emph{wedge} of $P$ over $K$. Observe that if $P$ is bounded or simple, then $W$ is too. Define the \emph{lower base} and \emph{upper base} of $W$ to be $P$ and $H\cap C$, respectively. Now for a vertex $x$ of $P$ not in $K$, define $x'$ to be $x\times (L \cap H$). Call $x'$ to be the vertex \emph{over} $x$ in relation to $W$, as would follow from our intuition of the natural projection of the upper base onto the lower base. Note that the lower and upper bases are combinatorially equivalent. The wedge clearly is a $d+1$-dimensional polytope, but what of its facets? Each facet of $P$ save $K$ is extended, so we there we have $n-1$ facets of $W$. We also have the lower and upper bases as facets, so in total, we have $n+1$ facets. Hence, $W$ is in the class $(d+1,f+1)$. See Figure~\ref{fig-wedge} for an illustration for a polytope $P$ in class $(2,4)$.

\begin{lem} The following is useful in our proof of Theorem~\ref{thm-hirschequiv}
\begin{enumerate}
\item A path, $l$, in a face $F$ of polytope $P$ is nonrevisiting if and only if $l$ is nonrevisiting in $P$.
\item A path, $l$ in the product of two polytopes $P_1$ and $P_2$ is nonrevisiting if and only the projections of $l$ in $P_1$ and $P_2$ are nonrevisiting in their respective polytopes.
\item If a path, $l$, in a wedge $W$ over $P$ is nonrevisiting, then the projection of $l$ onto $P$ is nonrevisiting. \citep{klee67}
\end{enumerate}
\end{lem}

\begin{figure}\label{fig-wedge}
\begin{center}
\begin{asy}
	import three;
	import patterns;
	add("tile",tile());
	currentprojection=orthographic(5,7,2,center=true);
	size(10cm);
	size3(3cm,5cm,8cm);
	pen bg=gray(0.9)+opacity(0.5);
	path3 p=(3,0,2)--(3,0,0)--(0,0,0)--(0,0,2)--cycle;
	draw((0,0,2)--(3,0,2)--(3,0,0)--(0,0,0));
	draw((0,0,0)--(0,0,2),red);
	draw(surface(p),bg,bg);
	path3 h=(6,6,2)--(6,6,0)--(-2,-2,0)--(-2,-2,2)--cycle;
	draw(surface(h),blue+opacity(0.5));
	draw((0,0,0)--(3,3,0)--(3,3,2)--(0,0,2));
	dot("$O$",(0,0,0),align=S);
	dot("$z$",(0,0,2),align=N);
	dot("$y$",(3,0,2),align=N);
	dot("$x$",(3,0,0),align=S);
	dot("$y'$",(3,3,2),align=N);
	dot("$x'$",(3,3,0),align=S);
	draw((3,0,0)--(3,4,0),Arrow3);
	draw((3,0,2)--(3,4,2),Arrow3);
	draw((0,0,2)--(0,4,2),Arrow3);
	draw((0,0,0)--(0,6,0),Arrow3);
	draw((0,0,0)--(0,0,2),red);
	label("$P$",(2.75,0,.5),NE);
	label("$K$",(0,0,1),E,red);
	label("$L$",(0,6,0),N);
	label("$H$",(6,6,.25),blue,align=NE);
\end{asy}
\caption{Wedge, $W$, of a 2 dimensional polytope, $P$, over its facet, $K$. $P$ is a quadrilateral, of which $K$ is an edge.}
\end{center}
\end{figure}

We are now ready to prove Theorem~\ref{thm-hirschequiv}.

\begin{proof}
From earlier we have that the nonrevisiting path conjecture implies Hirsch's conjecture. As the $d$-step conjecture is a special case of Hirsch's conjecture, the latter implies the former. All that is left is to prove that the $d$-step conjecture implies the nonrevisiting path conjecture, which is what we do now.

Suppose the $d$-step conjecture is true, that $\Delta(d,2d)\le d$. Then consider a bounded and simple triple $(P,x,y)$ in the class $(m,2m)$. Take $y_0=y$. Consider the face of smallest dimension such that $x$ and $y_0$ are incident to it. Call this face $P_0$. Then $(P_0,x,y_0)$ is a bounded and simple triple in the class $(d_0,d_0+m_0)$. Note that $m_0=d_0+k_0$, where $k_0$ is the number of facets of $P_0$ that are incident neither to $x$ nor $y_0$. If $k>0$, take $K_0$ to be a facet that is neither incident to $x$ nor $y_0$. Now let $P_1$ be the wedge of $P_0$ over $K_0$, and take $y_1$ to be the vertex over $y_0$ in relation to this wedge. We have that $(P_1,x,y_1)$ is a bounded and simple triple in the class $(d_1,d_1+m_0)$, where $d_1=d_0+1$. However, what is key is that $P_1$ has $k_1=k_0-1$ facets that are incident to neither $x$ and $y_1$, and none incident to both. Hence we iterate this process $k_0$ times until we get the bounded and simple triple $(P_{k_0},x,y_{k_0})$ in the class $(d_0+k_0,d_0+k_0+m_0) = (m_0,2m_0)$. 

By our construction, every facet of $(P_{k_0},x,y_{k_0})$ is incident to either $x$ or $y_{k_0}$ and none incident to both. $P_{k_0}$ is simple, so each vertex is adjacent to $m_0$ edges and facets, but there are $2m_0$ facets. A path from $x$ to $y_{k_0}$ then certainly visits every facet. Hence, such a path must be have at least $2m_0-m_0=m_0$ edges. However, we have assumed $\Delta(d,2d)\le d$, so there is a path from $x$ to $y_{k_0}$ with at most $m_0$ edges. Hence, this path has exactly $m_0$ edges, and, because $P_{k_0}$ is simple, this path must be nonrevisiting. We now project this path onto $P_{k_0-1}$ to get a nonrevisiting path from $x$ to $y_{k_0-1}$. Iterate until a nonrevisiting path from $x$ to $y$ in $P_0$.

As there exists a nonrevisiting path from $x$ to $y$ in $P_0$ a face of $P$, there exists a nonrevisiting path from $x$ to $y$ in $P$.
\end{proof}
\citet{klee67} conclude by disproving the general $d$-step conjecture for $d\ge 4$. They construct an example to illustrate that $\Delta_u(4,8)\ge 5$. Taking products of this counterexample, they demonstrate the following
\begin{prop}
(\citet{klee67}) $\Delta_u(d,n)\ge n-d+\min\{\lfloor d/4\rfloor,\lfloor(n-d)/4\rfloor\}$.
\end{prop}
Recently, \citet{santos10} gives us an improved lower bound for large $n$.
\begin{prop}
(\citet{santos10}) $\Delta_u(d,n)\ge (1+\epsilon)(d+n)$, where $d$ and $\epsilon$ are fixed and $n$ is arbitrarily large.
\end{prop}
The bounded Hirsch conjecture was disproven last year by Santos in the same paper \citep{santos10}. Santos constructs his counterexample by looking at a special class of polytope he calls \emph{spindles}. A \emph{$d$-spindle} is a $d$-polytope $P$ having two vertices $x$ and $y$ such that every facet of $P$ is incident to either $x$ or $y$ but not both. The \emph{length} of the spindle is the distance between $x$ and $y$. Note that in our proof of Theorem~\ref{thm-hirschequiv}, $(F_{k_0},x,y_{k_0})$ is an $m_0$-spindle. Santos proves the following $d$-step theorem for spindles:
\begin{thm}
{\rm \citet{santos10})} If $P$ is a spindle in the class $(d,n)$ and has length $l$, there exists another spindle, $P'$, in the class $(d-n,2d-2n)$ with length at least $l+n-2d$.
\end{thm}
Then disproving the bounded Hirsch conjecture amounts to finding a spindle whose length exceeds $d$. Santos does just this by finding a spindle in class $(5,48)$ of length 6. Hence, there is a spindle in class $(43,86)$ with diameter at least $44$, and the bounded Hirsch conjecture was proved false \citep{santos10}. Since the original counterexample, Santos has since found a counterexample of a smaller dimension: a polytope in class $(43,86)$ with $44$ vertices \citep{santos-slides}. Interestingly, Santos's counterexample is not a simple polytope; as we showed earlier, we can reduce the problem of maximum diameter of a polytope to maximum diameter of a simple polytope, but clearly here there was great value in thinking about the problem more generally.

Even more amazing, building on Santos's work, \citet{weib11} announced $23$ and $20$ dimension polytope counterexamples. Weibel's goal was to find a low dimension counterexample to Hirsch conjecture and to minimize the number of vertices of the polytope. He does this by considering the \emph{Minkowski sum} of particular polytopes. For polytopes $P_1$ and $P_2$, their Minkowski sum is $P_1+P_2:=\{x_1+x_2 : x_1\in P_1, x_2\in P_2\}$. More to the point, Weibel utilizes that when $P_1$ and $P_2$ have no facets parallel with the other, then any facet of $P_1$ or $P_2$ has a corresponding parallel facet of the same size and shape in $P_1+P_2$. He calls these facets \emph{pure facets}. Weibel claims that counterexamples can be built from any Minkowski sum in which the pure facets from different polytopes do not share any vertices in the sum polytope. He further claims that this can only happen when each of the summand polytopes has dimension at least $4$. 

\section{Bounds}
\begin{quote}
	\emph{You are a lover; borrow Cupid's wings,}\\
	\emph{And soar with them above a common bound.}\\
	--- Mercutio \citep{romeo}
\end{quote}
\subsection{Unrestricted Paths}
The result of the Hirsch conjecture being false is fascinating, but has little practical significance. We know that $n-d$ is not an upper bound for $\Delta$ or $\Delta_u$, but for all we know, the upper bound could still be linear in $n$ and $d$. The best know upper bounds are lacking.
\begin{lem}\label{lem-lar}
(\citet{lar70}) For any two vertices $x$ and $y$ in a $d$-dimensional polyhedron, there is a path that visits each face at most $2^{d-3}$ times .
\end{lem}
We do have $\Delta_u$ bounded linearly in $n$ and exponentially for $d$.
\begin{thm}
{\rm(\citet{lar70})} $\Delta_u(d,n)\le n2^{d-3}$.
\end{thm}
\begin{proof}
Let $x$ and $y$ be vertices on a $d$-dimensional polyhedron, and let $K$ be a path that begins with $x$ and ends with $y$ and visits each face of $P$ at most $2^{d-3}$ times. Such a path exists by Lemma~\ref{lem-lar}. For any two consecutive vertices in $K$, there is a facet that contains one but not the other, and conversely. Then, with each subsequent edge in $K$, there is a facet that has just been visited, and one that is newly visited. If the length of $K$ were greater than $n2^{d-3}$, there would then be at least one facet that was visited more than $2^{d-3}$ times, which is a contradiction. Hence, the path is less than $n2^{d-3}$, and because $x$ and $y$ were arbitrary, given $P$ was $d$-dimensional, we have $\Delta_u(d,n)\le n2^{d-3}$.
\end{proof}
For classes of polynomials in which $n$ and $d$ are close together, \citet{kalai92c} derived a better bound of $n^{\log_2(d)+2}$. However, we can sharpen their result to $n^{\log_{d}+1}$. 
\begin{lem}
\rm{(\citet{kalai92c})} $\Delta_u(d,n)\le\Delta_u(d-1,n-1)+2\Delta_u(d,\lfloor n/2\rfloor)+2$.
\end{lem}
\begin{proof}
Take $x$ and $y$ to be vertices on a polyhedron $P$ in class $(d,n)$. Let $k_x$ be the maximal number such that the union of the vertices in paths of length $k_x$ starting from $x$ is incident to no more than $n/2$ facets. Define $k_y$ similarly. There is a facet $F$ such that paths to $F$ from $x$ and $y$ are lengths $k_x+1$ and $k_y+1$.

We prove that $k_x\le\Delta_u(d,\lfloor n/2\rfloor)$. Create a polyhedron $Q$ from $P$ by deleting the inequality constraints that correspond to facets that cannot be reach from a path of length $k_x$ starting from $x$. So $Q$ is in class $(d,\lfloor n/2\rfloor)$ Consider a vertex $z$ in $P$ such that the distance from $x$ to $z$ in $P$ is $k_x$. Then the distance from $x$ to $z$ is $k_x$. We show this by supposing that the distance in $Q$ is less than $k_x$. Consider a path in which this is the case. Then there must be an edge that is not in $P$. Consider the first occurrence of such an edge, $E$, in the path. As $E$ is in $Q$ but not $P$, $E$ must intersect one of the hyperplanes, $H$, corresponding to a constraint we ignored to extract $Q$ from $P$. Then, in $P$, the distance from $x$ to $H$ is less than $k_x$. This is a contradiction, by our construction, the dsitance from $x$ to an ignored facet must be more than $k_x$. Hence, $k_x\le\Delta_u(d,\lfloor n/2\rfloor)$.

Then, the distance from $x$ to $F$ is bounded by $\Delta_u(d,\lfloor n/2\rfloor)+1$. The distance between any two vertices in $F$ is bounded by $\Delta_u(d-1,n-1)$. And, the distance from $F$ to $y$ is 
bounded by $\Delta_u(d,\lfloor n/2\rfloor)+1$. Hence we have the inequality, $\Delta_u(d,n)\le\Delta_u(d-1,n-1)+2\Delta_u(d,\lfloor n/2\rfloor)+2$, which gives us our result.
\end{proof}

Now we are prepared to prove a better bound. The following is entirely original work. At first, the author thought no one had derived this result, as the bound of $n^{\log_2(d)+2}$ is what is cited in the literature. However, since our initial derivation of the sharpened bound, $n^{\log_2(d)+1}$, we have found a paper from 2009 \citet{kim09} that used similar techniques to get the same result.
\begin{thm}
$\Delta_u(d,n)< n^{\log_2(d)+1}$.
\end{thm}
\begin{proof}
We have
$$\Delta_u(d,n)\le\Delta_u(d-1,n-1)+2\Delta_u(d,\lfloor n/2\rfloor)+2,$$
but since $\Delta$ is monotonic increasing in $d$ and $f$, we work with a cruder bound of
$$\Delta_u(d,n)\le\Delta_u(d-1,n)+2\Delta_u(d,\lfloor n/2\rfloor)+2,$$ Let's define
$$h(n,i):=\left\lfloor\left\lfloor\left\lfloor n/2\right\rfloor/\cdots\right\rfloor/2\right\rfloor,$$
where we have $i$ divisions by $2$.
We bound $\Delta_u(d,n)$ by a standard telescoping argument.
\begin{align*}
	\Delta_u(d,n)-2\Delta_u(d,\left\lfloor n/2\right\rfloor)&\le \Delta_u(d-1,n)+2\\
	2\Delta_u(d,\left\lfloor n/2\right\rfloor)-4\Delta_u(d,h(n,2))&\le 2\Delta_u(d-1,\left\lfloor n/2\right\rfloor)+4\\
	\vdots&\le\vdots\\
	2^{\left\lfloor\log_2\frac{n}{d}\right\rfloor}\left(\Delta_u(d,h(n,\left\lfloor\log_2\frac{n}{d}\right\rfloor))-2\Delta_u(d,h(n,\left\lfloor\log_2\frac{n}{d}\right\rfloor+1))\right)&\le 2^{\left\lfloor\log_2\frac{n}{d}\right\rfloor}(\Delta_u(d-1,h(n,\left\lfloor\log_2\frac{n}{d}\right\rfloor))+2)
\end{align*}
Hence we get
$$\Delta_u(d,n)\le\sum_{i=0}^{\log_2(n/d)} 2^i\left(\Delta_u(d-1,h(n,i))+2\right).$$
Before we continue, let us explain where some of the terms come from. We note that we have boundary conditions $\Delta_u(2,n)=n-2$ and $\Delta_u(d,n)=0$, the latter for $n\le d$. Then in the above telescoping inequalities, we have at most $\log_2(n/d)$ inequalities, as  $h(n,\lfloor\log_2 (n/d)\rfloor+1)\le d$, so $\Delta_u(d,h(n,\lfloor\log_2 (n/d)\rfloor+1))=0$.

Let
$$g(d,n):=(2d)^{\log_2 f}=f^{1+\log_2{d}}.$$
Note that the last equality comes from $x^{\log y}=y^{\log x}$. This holds because $\log(y^{\log(x)})=\log(x)\log(y)=\log(x^{\log(y)})$.

We claim that $\Delta_u(d,n)\le g(d,n)$. We proceed by strong induction. First we have that $\Delta_u(2,f)=f-2\le f^2=g(d,n)$, and $\Delta_u(n,f)=0\le f\cdot f^{\log_2 d}$, for $f\le d$. We induct on $d$,
\begin{align*}
\Delta_u(d,n)&\le\sum_{i=0}^{\log_2(n/d)} 2^i\left(\Delta_u(d-1,h(n,i))+2\right)\\
&\le\sum_{i=0}^{\log_2(n/d)} 2^i \left(g(d-1,h(n,i))+2\right)\\
&\le\sum_{i=0}^{\log_2(n/d)} 2^i \left(g(d-1,n/2^i)+2\right)\\
&=\sum_{i=0}^{\log_2(n/d)} 2^i (2(d-1))^{\log_2 n/2^i}+2\sum_{i=0}^{\log_2(n/d)}2^i \\
&=f(d-1)^{\log_2 f}\sum_{i=0}^{\log_2(n/d)} \left(\frac{1}{d-1}\right)^i+2(2\cdot\frac{n}{d}-1)\\
&<f(d-1)^{\log_2 f}\cdot\frac{d-1}{d-2}+\frac{4f-2d}{d}\\
&=f(d-1)^{\log_2 f}\cdot\frac{d-1}{d-2}+\frac{4f-2d}{d}.
\end{align*}
We claim that this final expression is less than $g(d,n)$. To see this, note that
$$\frac{(d-1)^{1+\log_2 n}}{d^{1+\log_2 n}+2d/n-4}<\frac{d-2}{d}, \text{ for }d\ge 3,n>d.$$
This implies that
$$nd(d-1)^{1+\log_2 n}<(d-2)\left(nd^{1+\log_2 n}+2d-4n\right).$$
Finally, we have
$$n(d-1)^{\log_2 n}\cdot\frac{d-1}{d-2}+\frac{4n-2d}{d}<nd^{\log_2 n}=n^{1+\log_2 d}=g(d,n).$$
This completes the inductive step, and we thus have, $\Delta_u(d,n)<n^{\log_2(d)+1}$.
\end{proof}
\subsection{Monotone Paths}
We consider now a slightly different question, one that more directly tied to linear programming.
\begin{conj}
(Monotone Hirsch conjecture) If $\phi$ is a linear form on a polyhedron $P$, then each vertex of $P$ can be joined by an monotone increasing nonrevisiting path to an optimal vertex.
\end{conj}
The conjecture is a more restrictive form of Hirsch's conjecture. The conjecture says that for a vertex $x$ in $P$ there is a nonrevisiting path to $x_{\max}$ in $P$ where $x_{\max}$ maximizes $cx$ over $P$. As we showed earlier, a nonrevisiting path is equivalent to the diameter being at most $n-d$. The monotone Hirsch conjecture was shown to be true for 3-dimensional polyhedra by \citet{klee65}, but was eventually disproved by \citet{todd80}. Todd constructs a 4-dimensional polytope with 8 facets in which every monotone path is at least length 5. This idea of monotone paths is certainly interesting, as it seems closer to the realization of a deterministic pivot rule.

Take $H(d,n)$ and $H_u(d,n)$ to be smallest number of edges needed for a monotone increasing path from any vertex $x$ to $x_{\max}$ for bounded and unbounded polyhedra, respectively. \citet{todd80} demonstrates that $n-d+\min\{\lfloor d/4\rfloor,\lfloor(n-d)/4\rfloor\}\le H(d,n)\le H_u(d,n)$.

We even have a bound on $H_u$, thus giving us a pseudopolynomial monotone path to the optimal vertex.
\begin{thm}
\rm{(\citet{kalai92}, \citet{zieg94})} $H_u(d,n)\le 2f^{1+\log_2(d)}=2(2d)^{\log_2{n}}$.
\end{thm}
\begin{proof}
The argument hinges on the idea of an \emph{active facet}, which we define here. For a vertex $v$ of a polyhedron $P$, and a linear function $cx$, a facet of $P$ is \emph{active} if it contains a point that is higher than $v$; that is, either the facet is unbounded with respect to $cx$ or it has a top vertex $w$ such that $cv<cw$. To be clear about which vertex and polyhedron we are considering a facet to be \emph{active}, we write $F$ is an \emph{active facet} of $P(v)$.

Define $\bar{H}(d,n)$ to be the length of a path required from a vertex $v$ that has at most $f$ active facets and an arbitrary number of nonactive facets. Then $H(d,n)\le \bar{H}(d,n)$.

We can easily derive exact values for $\bar{H}$ on boundaries. We have $$\bar{H}(2,f)=f,$$ as all edges in the path to the optimal vertex are active, and we may have to traverse all of them. Effectively, we are adding vertices to unbounded edges here, starting along one unbounded edge, and finishing along the other. We continue to consider vertices on the unbounded edges throughout the rest of this proof. Next, we can see that $$\bar{H}(d,0)=\ldots=\bar{H}(d,d-2)=0,$$ as if a vertex is not optimal, there is an edge that is increasing; this edge lies on $d-1$ facets, all of which are necessarily active.

Now we want to derive a recursion for $\bar{H}(d,n)$. We do this using four key claims.

\emph{Given any set $\mathcal{F}$ of $k$ active facets of $P(v)$, there is a monotone path from $v$ to the optimal vertex of $P$ or some vertex in some facet of $\mathcal{F}$ that has length at most $\bar{H}(d,n-k)$.}

If $v$ lies on a facet of in $\mathcal{F}$, we are immediately finished. Consider then when $v$ does not lie on a facet in $\mathcal{F}$. Let $P=P(A,b)$ be a minimal system in that we do not have any redundant constraints; then each constraint defines a facet of $P$. Now take $P'$ to be the polyhedron defined by deleting constraints that correspond to facets in $\mathcal{F}$. Then $v$ is a vertex of $P'$ and then has at most $n-k$ active facets in $P'(v)$. Now consider the shortest monotone path in $P'$ from $v$ to the optimal vertex in $P'$. Such a path has length at most $\bar{H}(d,n-k)$. Take this path back in $P$, to get that a monotone path of length at most $\bar{H}(d,n-k)$ from $v$ must either reach or not reach a facet in $\mathcal{F}$. If it does reach, which happens when the path in $P'$ enters an edge that was not in $P$, our claim is complete. If it does not reach, then the optimal vertex in $P'$ is in fact the optimal vertex in $P$; this path in $P'$ is the same as the path in $P$, as else we hit a facet in $\mathcal{F}$, hence our claim is complete.

\emph{The collection $\mathcal{G}$ of all active facets for which there is a path from $v$ with length at most $\bar{H}(d,n-k)$ has at least $n-k+1$ active facets.}

Note that there are ${n\choose k}$ total sets of $k$ active facets. Since there is at least one facet, say $F_1$ within $\bar{H}(d,n-k)$, we can add $F_1$ to $\mathcal{G}$ and note there are now ${n-1\choose k}$ sets of $k$ active facets that do not include $F_1$. Continuing in this manner until we have at least $\{F_1,\ldots,F_k\}\subset\mathcal{G}$. There is ${k \choose k}=1$ subset which does not contain a facet in $\{F_1,\ldots,F_k\}$. Again, this subset contains an active facet within $\bar{H}(d,n-k)$, so we have $\{F_1,\ldots,F_{k+1}\}$, as desired.

\emph{There exists a monotone path from $v$ to the optimal vertex $w_0$ in $\mathcal{G}$ with length at most $\bar{H}(d,n-k)+\bar{H}(d-1,n-1)$.}

This follows immediately from above. It takes at most $\bar{H}(d,n-k)$ edges to reach the active facet $w_0$ is on; then, because the facet is $d-1$-dimensional and has at most $n-1$ facets, it takes at most $H(d-1,n-1)$ edges to get to $w_0$.

\emph{There is a monotone path from $w_0$ to the optimal vertex with length at most $\bar{H}(d,k-1)$.}

Since we are at the optimal vertex in $\mathcal{G}$, none of the facets in $\mathcal{G}$ is active for $P(w_0)$. Then, $w_0$ has at most $n-(n-k+1)=k-1$ active facets. Hence, the monotone path from $w_0$ to the optimal vertex has length at most $\bar{H}(d,k-1)$, as desired.

Collecting the inequalities, we have
$$\bar{H}(d,n)\le\bar{H}(d,n-k)+\bar{H}(d-1,n-1)+\bar{H}(d,k-1)$$
for any $k$.

To reduce this inequality, we make use of the fact that $\bar{H}$ is increasing in $f$ and take $k=\lceil n/2\rceil$. Hence,
$$\bar{H}(d,n)\le\bar{H}(d-1,n-1)+2\bar{H}(d,\lfloor n/2\rfloor).$$

Take
$$g(d,t):=2^{-t}\bar{H}(d,2^t), \text{for $t\ge 0$ and }d\ge 2.$$
Then we have
\begin{align*}
	g(d,t)&=2^{-t}\bar{H}(d,2^t)\\
	&\le 2^{-t}(H(d-1,2^t-1)+2\bar{H}(d,\lfloor 2^{t-1}\rfloor))\\
	&=2^{-t}H(d-1,2^t-1)+2^{-(t-1)}\bar{H}(d,2^{t-1}))\\
	&\le g(d-1,t)+g(d,t-1).
\end{align*}
Transforming the boundary conditions, we have
\begin{align*}
	g(2,t)&=1\le{t \choose 1} \text{ for }t\ge 1\\
	g(d,0)&=0={d-2\choose d-1}\text{ for }d\ge 3,
\end{align*}
from which we get, by inducting on $t\ge 0$ and $d\ge 2$,
$$g(d,t)\le{d+t-2\choose d-1}.$$
Finally, we get
\begin{align*}
	\bar{H}(d,n)&\le\bar{H}(d,2^{1+\lfloor\log_2 n\rfloor})\\
	&=2^{1+\lfloor\log_2 n\rfloor}g(d,1+\lfloor\log_2 n\rfloor)\\
	&\le 2n{d+\lfloor\log_2 n\rfloor-1\choose d-1}\\
	&\le 2nd^{\log_2 f}\\
	&=2n^{1+\log_2 d}
\end{align*}
for $n,d\ge 2$. The last inequality follows by ${a+b\choose b}\le(a+1)^b$, by inducting over $a,b\ge 0$.

Thus, we have $H_u(d,n)\le\bar{H}(d,n)\le 2(2d)^{\log_2 n}$.
\end{proof}

\section{Special Polytopes}
\begin{quote}
	\emph{Nous sommes tous des cas exceptionnels. Nous voulons tous faire appel de quelque chose!}\\
	--- Albert Camus, La Chute
\end{quote}
Now we restrict our consideration to special classes of polytopes. We look at the maximal diameter of these polytopes, and in doing so, whether they satisfy the Hirsch conjecture, and in some cases, discuss how the simplex algorithm performs in these special circumstances.
\subsection{(0,1)-Polytopes}
Let $P$ be a (0,1)-polytope, such that $P$ is the convex hull of $V\subseteq\{0,1\}^d$. \citet{nad89} proved the Hirsch conjecture for $(0,1)$-polytopes. We have simplified the proof here a little by requiring less machinery (in fact, no heavy machinery at all) in the following lemma.
\begin{figure}
\begin{center}
\begin{asy}
	size(14cm);
	pen bg=gray(0.9)+opacity(0.75);
	draw((0,0)--(0,1)--(1,1)--(1,0)--cycle);
	filldraw((0,0)--(1,0)--(1,1)--cycle,bg,black);
	dot((0,0)); dot((1,0)); dot((1,1));
	draw((-2,0)--(-2,1)--(-1,1)--(-1,0)--cycle);
	draw((-2,0)--(-1,1));
	dot((-2,0)); dot((-1,1));
	filldraw((2,0)--(2,1)--(3,1)--(3,0)--cycle,bg,black);
	dot((2,0)); dot((2,1)); dot((3,1)); dot((3,0));
\end{asy}
\caption{The (0,1)-polytopes in dimension $2$, up to isomorphism}
\end{center}
\end{figure}
\begin{lem}
$\delta(P)\le \dim P$. We have equality when $P$ is isomorphic to the cube.
\end{lem}
\begin{proof}
We proceed by induction on $d$. This is certainly true for $d=2$. Now consider two vertices $u$ and $v$ of $P$ such that $\delta(u,v)\ge d$. By symmetry of the cube, we can assume that $u=0$ and $v\in\{0,1\}^d$. Further, we can assume that $\dim P=d$, as otherwise we can project $P$ into a lower dimension space, to be easier, into $\{0,1\}^{d-1}$, and by induction we have our claim. So $\dim P=d.$

Suppose now that $v_i=0$ for some $i$. Then $u=0$ and $v$ share a facet, specifically, the facet defined by $x_i\ge 0$. Thus, we get $\delta(u,v)=\delta(0,v)\le d-1$, by the inductive hypothesis. Hence, we now consider the case when $v_i\not=0$, for any $i$. That is, $v=1$. Now, by induction we can show that any vertex adjacent to $v=1$, say $w$, that has $k$ 0-components have $\delta(0,w)\le k$. From this we get, $(u,v)=(0,1)\le(0,w)+(w,1)\le d-k +1$.

So if $w$ has $k>1$ 0-components, we have $\delta(u,v)<d$. Hence, if $\delta(0,1)\ge d$, then all vertices adjacent to $1$ have only one $0$-component. But $1$ has at least $d$ adjacent vertices; so set of adjacent vertex is in fact those with exactly one 0-component. By induction on $d$, these vertices are distance $d-1$ from $0$ in the facet they share with $0$. This facet though is the convex hull of points such that the respective component is $0$. That is, the facet is a $d-1$ cube. Assembling these $d-1$-cubes, we can see that the convex hull of the points in $P$ form the $d$-cube, and that $\delta(0,1)=d$, as desired.
\end{proof}
\begin{thm}
	The Hirsch conjecture holds for (0,1)-polytopes.
\end{thm}
\begin{proof}
Again, we induct on the dimension of $P$. We can see the hypothesis holds for $\dim P=2$. Assume the hypothesis holds for $\dim P=d$. Now take $P$ such that $\dim P = d+1$. Take $u$ and $v$ such that their distance is the diameter of $P$. Then there are at least $d+1$ facets incident to $u$ and $d+1$ facets incident to $v$. Suppose there is a facet incident to both. Then by the inductive hypothesis, we are finished. Then consider if there is not at least one facet incident to both $u$ and $v$. Then there are at least $2(d+1)$ facets and $P$ is $d+1$ dimensional, so $n-d=d+1\ge \delta(u,v)$, by the above lemma.
\end{proof}
Despite $(0,1)$-polytopes satisfaction of the Hirsch conjecture, a polynomial deterministic simplex algorithm has not yet been found.

\citet{klein92} generalize a bound for the diameter of $\{0,\ldots,k\}^d$ polytopes.
\begin{thm}
	If $V\subseteq \{0,\ldots,k\}^d$, then the diameter of the convex hull of $V$ is at most $k$ times the dimension of $V$.
\end{thm}

\subsection{Network Flow Polytopes}

We specifically consider the minimum cost flow problems of transportation and assignment. The transportation polytope is defined as

$$T_{m,n}(a,b):=\{x\in\mathbb{R}^{m\times n}:\sum_i^m x_{i,j}=b_j, \sum_j^n x_{i,j}=a_j, x_{i,j}\ge 0\}.$$

\citet{bright06} show that the diameter of any $m\times n$ transportation polytope is at most $8(m+n-2)$. \citet{kim09} cites the following bound received by correspondence. The author of this paper has been unable to find a paper or pre-print to verify this.

\begin{thm}
\rm{(\citet{hurk07})} The diameter of any $m\times n$ transportation polytope is at most $3(m+n-1)$
\end{thm}

\citet{bal84} considers the dual of the transportation problem and demonstrates the following:
\begin{thm}{\rm (\citet{bal84})} The diameter of the dual $m\times n$ transportation polytope is at most $(m-1)(n-1)$. This bound is optimal.
\end{thm}

As the dual transportation polytope is in class $(m+n-1, mn)$, this proves the Hirsch conjecture for dual transportation polytopes. In \citep{bal84}, Balinski establishes a combinatorial characterization of the polytopes vertices; he calls this characterization a \emph{signature}. We mention this because through this characterization, he makes progress with the simplex algorithm on the assignment problem in \citet{bal85}.

Recall the assignment problem: We are given a set of $n$ people and a set of $n$ tasks and a cost $c_{i,j}$ for the $i$th person to conduct the $j$th task.  Let $f_{i,j}$ be an indicator function, taking $1$ if the $i$th person does task $j$, and $0$ otherwise. Then we wish to minimize $\sum_i^n\sum_j^n c_{i,j}f_{i,j}$, subject to $\sum_i^n f_{i,j}=1$ and $\sum_j^n f_{i,j}=1$, that is, subject to a person taking on one and only one task. The $n\times n$ polytope, $P_n$, is derived from the assignment problem. $P_n:=T_{n,n}(1,1)$. As we mentioned earlier, \citet{bal85} derived a result for the simplex algorithm on the assignment problem.

\begin{thm}{\rm (\citet{bal85})}
	There is a simplex implementation on the assignment problem whose worst-case behavior terminates in $n-1 \choose 2$ pivots.
\end{thm}

This is interesting as this shows that the assignment problem is one of the few times in which the simplex algorithm is competitive with non-simplex procedures in worst case behavior. We now go back to considering the geometry of the assignment polytope.

\citet{bal74} first demonstrated that the assignment polytope satisfies the Hirsch conjecture by demonstrating that it has diameter $2$. We give a simpler proof of this fact using permutations.

\begin{thm}
	The diameter of $P_n$ is 2.
\end{thm} 
\begin{proof}
First note that there are $n!$ ways to arrange the tasks amongst the people, so $P_n$ has $n!$ vertices. This inspires us to consider a correspondence between the vertices and the set of permutations on $m$ distinct letters. Recall from introductory algebra that any permutation can be decomposed into a product of disjoint cycles. Say we have a permutation $$(j_1,\ldots,j_n)=(j_1,\ldots,j_{n_1})(j_{n_1+1},\ldots,j_{n_2})\cdots(j_{n_k}+1,\ldots,j_n),$$ then note that we can express the permutation as the product of two cycles $$(j_1,\ldots,j_n)=(j_1,\ldots,j_n)(j_n,j_{n_k},\ldots,j_{n_1}).$$
Let $Q'(n)$ be the set of permutations on $m$ letters that can be expressed in cycle notation by a single cycle. Let us further take $p$ to map the natural correspondence between vertices of $P$ to permutations on $m$ letters. That is, $p$ maps permutation matrices to their natural permutation. It can be shown that vertices $x$ and $y$ of $P_n$ are neighbors if and only if there is a $z\in Q'(m)$ such that $p(x)=p(y)p(z)$, \citep{bal74} and \citep{young78}. Hence the diameter of $P_n$ is greater than $1$, for $n\ge 4$.

Consider any two vertices, $x$ and $y$, in $P$. We have from above that there exists $w,v\in Q(n)$ such that $p(x)^{-1}p(y)=p(u)p(v)$. Supposing $u$ is not the identity, $y$ then is a neighbor of the vertex corresponding to $p(x)p(u)$, say $w$. We then immediately have that $w$ is a neighbor of $x$, as $p(w)=p(x)p(u)$. So $x$ and $y$ are distance at most $2$ apart. Hence the diameter of $P_n$ is $2$, for $n\ge 4$.
\end{proof}
\subsection{Leontief substitution systems}
A Leontief matrix, $A\in\mathbb{R}^{m\times n}$, has rank $m$ and exactly one positive element in each column. A pre-Leontief matrix, $M\in\mathbb{R}^{m\times n}$, has rank $m$ and has at most one positive element in each column. $P=\{x : Ax=b,x\ge 0\}$, where $b$ is a constant positive $m$-vector, is a Leontief substitution system, and $Q=\{x : Mx=b,x\ge 0\}$, where $b$ is a constant non-negative $m$-vector, is a pre-Leontief substitution system.

Not only is the Hirsch conjecture true for pre-Leontief systems (and hence Leontief systems), but the monotone Hirsch conjecture is true for pre-Leontief systems. In fact, the bound we can achieve is slightly lower than that of the monotone Hirsch conjecture.

\begin{thm}
{\rm(\citet{grin71})} Take $x$ and $y$ to be vertices of the pre-Leontief substitution system, $Q$. If both $x_i$ and $y_i$ are positive for $k$ columns in $B$, then the distance between $x$ and $y$ is at most $m-k$. Further, if $y$ is the optimal vertex, a path can be chosen from $x$ to $y$ such that the objective function is nonincreasing.
\end{thm}
Note that the distance between any two vertices is at most $m-k$. As $k\ge 1$, we immediately have Hirsch's conjecture satisfied. Also note that the shortest route polytope is a pre-Leontief system, so while it was previously shown by \citet{rom69} that the shortest route problem satisfies the Hirsch conjecture, we now get it for free.
\begin{proof}
First we show the claim holds for Leontief systems. Let $\mathcal{B}'$ be the set of all $m\times m$ Leontief submatrices of $A$. Consider $B\in \mathcal{B}$. Take $B-$ to be the submatrix of $A$ with the columns of $B$ deleted. Denote $x_B$ to be the $m$-vector made of the components of $x$ in which the columns of $A$ appear in $B$. By \citet{vein68}, we say that $x$, for which $Ax=b$, is \emph{determined} by a matrix $B$ if $B$ is a basis of $A$ and $(x_B,x_B-)=(B^{-1}b,0)$. This is a little confusing, so we give an example. Take	
\begin{center}
$A=
\left(
\begin{array}{ccc}
	1 & 0 & -1\\
	0 & 1 & 1
\end{array}
\right),\,
b=
\left(
\begin{array}{c}
	1 \\
	1
\end{array}
\right),$ and
$B=
\left(
\begin{array}{cc}
	1 & -1\\
	0 & 1
\end{array}
\right)$.
\end{center}
We thus get
\begin{center}
$x_B=
\left(
\begin{array}{c}
	2\\
	1
\end{array}
\right)$
and the determined
$x=
\left(
\begin{array}{c}
	2\\
	0\\
	1
\end{array}
\right)
$.
\end{center}

By \citet{vein68}, $x\in P$ is a vertex if and only if $x$ is determined by a $B\in \mathcal{B}$.

We assume that $y$ is the optimal vertex for the linear program $\min\{ cx: Ax=b,x\ge 0\}$. As $y$ is a vertex, we have that there is some $D\in \mathcal{B}$ that determines $y$, and since $D$ is nonnegative, so is its inverse, which implies $D$ maintains its optimality as a basis no matter what values $b$ take, so long as $b$ is nonnegative.

Now take $B_0\in \mathcal{B}$ such that $B_0$ determines $x$. $B_0$ shares at least $l\ge k$ columns with $D$. We now try to construct an $m-l+1$ sequence of submatrices of $A$, that is, bases, $B_0,\ldots, B_{m-l}$. We hope to construct the sequence in such a way that $D=B_{m-1}$, adjacent bases in the sequence share $m-1$ columns in common, and the objective function is nonincreasing along the sequence of vertices determined by the bases. If we can do this, we have our result.

We have three cases: first, the nondegenerate case in which $b$ is strictly positive and $y$ is the unique optimal vertex, next the degenerate case in which $b$ has some 0 elements, and finally, the degenerate case in which there are multiple optimal vertices.

First we take care of the nondegenerate case, so assume that $b$ is strictly positive and $y$ is the unique optimal vertex. That $b$ is strictly positive implies there is a bijection between vertices and bases and that each basis is in $\mathcal{B}$. Consider a basis $B_j\in \mathcal{B}$. We then have that the dual variables corresponding with this basis are $c_{B_j}(B_j)^{-1}$, and the value of the objective function with this basis is $c_{B_j}(B_j)^{-1}b$. Now say there is a column of $D$, $d_i$, with cost $c_i$, that is eligible to enter the current basis, $B_j$. That is, $c_i<c_{B_j}(B_j)^{-1}b$. If we swap $d_i$ with a column in $B_j$, we must ensure the new basis corresponds to a vertex. The only way to do that is, as we mentioned earlier, force the basis to be in $\mathcal{B}$. The only way to do that is to force the swap to be with the column of $B_j$ that has its positive element in the same row as $d_i$'s positive element. We conduct the swap and obtain the new basis $B_{j+1}$. Note that $B_{j+1}$ shares one more column in common with $D$ and its dual variables cannot be greater that those of $B_j$, hence the objective function can only decrease from the change in basis. If now column of $D$ was eligible to enter the basis, the current basis is optimal, as we have assumed $D$ is the optimal solution. Hence, we have built the desired sequence of bases, and the claim holds for the nondegenerate case.

Consider the degenerate case in which $b$ is not strictly positive. Consider the same problem but replace $b$ with $b'>0$. The inverse of every $B\in \mathcal{B}$ is nonnegative, so the sequence as we constructed above is feasible for any nonnegative $b$. However, we lose that the objective function is strictly decreasing along the path; instead we satisfy ourselves that it is nonincreasing along the path.

Finally, consider the degenerate case in which $y$ is not the optimal solution. This is the easiest to deal with, as we use the original argument of the nondegenerate case. In the event we enter a basis that is also optimal before we enter the basis that determines $y$, we terminate the sequence, hence obtaining a sequence that is strictly less than that proposed. Thus, our sequence of bases has still at most $m-l+1$ bases.

We have exhausted the result for Leontief substitution systems. Let us move to pre-Leontief substitution systems. We can reduce the pre-Leontief case to the Leontief case by making heavy use of results found in \citet{vein68}.

Suppose we have a pre-Leontief system, then we can partition $M$ and $b$ such that

$$M=
\left(
\begin{array}{cc}
	A_1, & A_3\\
	0, & A_2
\end{array}
\right),
b=
\left(
\begin{array}{c}
b^1\\
b^2
\end{array}
\right)
$$
where $A_1$ is Leontief and $A_2$ is sub-Leontief (the definition of which is not important for this discussion, but if interested, do see \citep{vein68}), and where $b^i$ is permuted to correspond with the form of presented for $A$. Partitioning $x$ to $x=(x^1,x^2)$, we see that $Q^1=\{x^1: A_1 x^1 =b^1, x^1\ge 0\}$ is a Leontief substitution system. Our original system, $Q=\{x: Mx=b,x\ge 0\}$ is non empty if and only if $b^2=0$. Further, $x$ is a vertex of $Q$ if and only if $x^1$ is a vertex of $Q^1$ and $x^2=0$. Hence, for any vertices in $Q$, we map to vertices in the corresponding pre-Leontief system, which we have already shown holds for our claim. Hence we have shown the result for pre-Leontief substitution systems, so our claim is complete.
\end{proof}

A \emph{Leontief flow problem} is the linear programming problem
$$\min\{ cx : Ax=b, x\ge 0\}$$
where $A$ is an integral Leontief matrix with all positive entries equal to $1$ and the cost vector $c$ is integral as well.

\citet{ng96} give a result for the simplex algorithm's performance on Leontief flow problems.

\begin{thm}
{\rm (\citep{ng96})}
	The simplex algorithm, using Dantzig's pivot rule, solves the Leontief flow problem in $\mathcal{O}(n^2 U \log (npU))$ pivots, where $p$ is the entry of largest magnitude in the Leontief matrix $A\in\mathbb{R}^{m\times n}$, and $U$ is an upper bound on any variable in $x$ in any solution on a vertex.
\end{thm}

In general this bound is not polynomial, as $U$, for example, $U$ can easily be exponential. However, there are some cases, such as the shortest-path problem, in which $U=m$ and $p=1$. For the shortest-path problem, however, \citet{orlin85} gives a better bound of $\mathcal{O}(m^2 n\log m)$ pivots.

\section{Concluding thoughts}
\begin{quote}
\emph{Not with a bang but a whimper.}\\
	--- T.S. Eliot, The Hollow Men
\end{quote}

We finish off with a consideration of what ``average case'' behavior may mean for the simplex algorithm and discuss some recent results and developments.

\subsection{Random Polytopes}
Let's break from special polytopes and think about why the simplex algorithm may work well generally. A natural idea to explore this is to set our constraints to be random variables under nice or natural distributions. This is exactly what we do.

For the linear program we originally introduced, $\text{(P): minimize}\left\{ c^T x : Ax= b, x\ge 0\right\}$, we now take entries of $A, b,$ and $c$ to be entries of random variables. We hope that we can bound the number of expected pivots for a given distribution and implementation of the simplex.

Here, we state results of the average behavior of the simplex algorithm over random polytopes and discuss some of the ideas used in the analysis.

We first consider the work done by \citet{smale83}. In the paper, Smale proposes a spherically symmetric probabilistic model for analysis of the seln-dual simplex. This model is equivalent to taking the coefficient random variables to be standard normal distributed \citet{meg86}. Let $\rho(m,n)$ be the expected number of pivots required of the seln-dual simplex when $A$ is an $m\times n$ matrix. Smale demonstrates that $\rho(m,n)\le c(m)\log n^{m(m+1)}$, where $c(m)$ is a constant dependent on $m$. 

What is key is that Smale shows that we can express $\rho$ as the sum of volumes of four different types of matrices, each of which correspond to different types of bases. The volume of a matrix is its determinant, but a notion the author had not encountered that is made use of is the \emph{Gaussian volume} of a matrix, $GV(A)$ for $A\in\mathbb{R}^{m\times n}$. $GV(M)$ is the probability that a random vector distributed standard $n$-normal lines in the convex cone spanned by $M$'s columns. This is relevant because Smale expresses $\rho $ in terms of probabilities of random rays being in specific random cones.

Other work in this area was done independently by \citet{borg82}. Borgwardt worked under a different probabilistic model than did Smale, and under this model Borgwardt tests a special variant of the simplex of his design. Under this model, Borgwardt proves a bound for expected pivots $\phi(m,n)\le cmn^2(n+1)^2$, where $c$ is a constant equal to $(e\pi/4)(\pi/2+1/e)$.

\subsection{Recent Work}

The recent IPAM conference ``Efficiency of the Simplex Method: Quo vadis Hirsch conjecture?'' and Gil Kali's proposal of the polynomial Hirsch conjecture as the Polymath3 project have resulted in a slew of results and ideas in the past few months. Here we cover a few.

\subsubsection{Subexponential lower bounds for random pivot rules}

At the IPAM conference, subexponential lower bounds of the form $2^{n^\alpha}$ were proved for two basic randomized pivot rules for the simplex \citet{free11}. The two pivot rules they consider are {\sc Random-Edge} and {\sc Random-Face}. The former is the most natural random pivot rule, while the latter is theoretically the fastest pivot rule currently known. Before this result, no non-polynomial lower bounds existed for the rules' performance, and finding such bounds was an open problem for decades. 

\citet{free11} construct concrete linear programs for which they obtain expected subexponential number of pivots. For {\sc Random-Edge}, they construct \emph{Parity games}, a family of deterministic 2-player games, for which {\sc Random-Edge} takes an expected subexponential number of pivots. They transform the parity games into a Markov Decision Processes, a family of stochastic 1-player games. Finally, they transform the Markov Decision Process into linear programs. For {\sc Random-Facet}, the process is the same, but the construction differs. The key idea lies in the construction themselves, and thinking about the problem in terms of parity games. As the authors note, had they considered Markov Decision Process constructions directly, they likely would not have succeeded.

\subsubsection{Polymath3}

The polynomial Hirsch conjecture, for a polytope in class $(d,n)$, $\Delta$ is bounded above by a polynomial in $d$ and $f$, was chosen as the Polymath3 project last September. The approach to tackle a combinatorial formulation drew great interest. We present the combinatorial abstraction given here:

\begin{conj}
(Combinatorial polynomial Hirsch conjecture) Take $t$ non-empty disjoint families, $F_1,\ldots,F_t$, of subsets of the first $n$ naturals, $\{1,\ldots, n\}$. We place the further restriction on the families, that for all $i<j<k$, for any $S\subset F_i$ and any $T\subset F_k$ we have some $R\subset F_j$ such that $S\cap T\subset R$. Take $f(n)$ to be the largest value of $t$ for the input $n$. The conjecture is that $f(n)$ is polynomial in $n$.
\end{conj}

The combinatorial polynomial Hirsch conjecture implies the polynomial Hirsch conjecture.
\begin{thm}
A simple polytope with $n$ facets has diameter bounded by $f(n)$.
\end{thm}
The result is given by \citet{eisen10}, but the proof we give is by Kalai. We notice the proof is similar in flavor to the proof of Kalai and Kleitman's bound.
\begin{proof}
Let $P$ be a simple polytope with $n$ facets. Let $x$ be a vertex of $P$. Define $S_x$ to be the set of facets that contain $x$. Choose a vertex $w$ such that there exists a vertex $v$ whose distance from $w$ is the diameter of $P$. Now define $F_i$ to be the family of sets that correspond to vertices that are distance $i+1$ from $w$. That is, $F_i$ contains $S_x$'s, not vertices. Then the number of families is exactly the diameter of $P$. Hence, why we seek to maximize the number of families in the formulation.

Now we show why we have the further restriction. Take vertices $v$ and $u$ such that $v$ is distance $i$ from $w$ and $u$ is distance $k>i$ from $w$. Now take the shortest path from $v$ to $u$ in the smallest face containing both $u$ and $v$. For every vertex $z$ in this path we have that $S_u\cap S_v\subset S_z$. Now note in the path, the distance of adjacent vertices to $w$ can change at most by $1$, so there is some vertex in the path that is distance $j$, for $i<j<k$, away from $w$. Hence, we have showed the claim.
\end{proof}

Other abstractions and continued discussions for the interested reader can be found on Gil Kalai's blog\footnote{As of the middle of April, Kalai has been trying a topological abstraction, inspired by Santos's work in the counterexample to the Hirsch conjecture. Since our review of the subject is concerned with connections to the Simplex algorithm, and is entirely combinatorial, we do not include this very recent line of pursuit in this overview. While interest in the topologic formulation is not as great as that for the combinatorial formulation--the latter had active participation and discussion from renowned combinatorialists--that it uses insights from Santos's counterexample and is very elegant demonstrates promise.} \citep{kalaiblog}. Also, a talk was given at IPAM on abstractions of the polynomial Hirsch conjecture. See \citet{hahn10} for the presentation.

\subsection{Further Directions}

When Gil Kalai posted on his blog that Francisco Santos had disproved the Hirsch conjecture, there were a great many enthusiastic responses. However, amongst the happy congratulators there were a few dismayed doctoral students,

\begin{quote}
``That's my whole PhD work going to trash!'' \\
--- Tarantino from Stanford
\end{quote}
\begin{quote}
``I was working on the same problem. I guess I need to find a new research topic.'' \\
--- A PhD student from CMU
\end{quote}
drawing responses from Santos and Kalai,
\begin{quote}
``To Tarantino: I hope that is not true. As Anand says, I think the overall question `how much can the diameter of a polytope grow' is as interesting now as it was before.''\\
--- Francisco Santos
\end{quote}
\begin{quote}
	``To the Stanford and CMU PhD srudents, [sic]\\
	While this is certainly a major development, there are quite a few remaining interesting questions in this research area, and probably reading Santos's paper will raise additional problems and ideas.''\\
	--- Gil Kalai
\end{quote}
The main reason of highlighting the exchange is to elicit amusement from the reader, but it also addresses where to go next. Despite the Hirsch conjecture being disproven, it is still possible that the upper bound on a polyhedron's diameter is linear. Fruitful results are more likely to be found in exploring the polynomial Hirsch conjecture, for which a combinatorial reformulation has already produced many ideas. Reformulation is a viable attack on bounds for expected iterations of simplex algorithm, as well. Simply, the bounds we have for polyhedron diameter and expected iterations of the simplex are gigantic. Any refinement or reformulation of either problem, or insight in how these functions grow is still relevant and of theoretical and practical importance.

\bibliographystyle{plainnat}
\bibliography{essay}


\end{document}